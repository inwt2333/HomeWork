\documentclass[12pt, a4paper, oneside]{ctexart}
\usepackage{amsmath, amsthm, amssymb, bm, color, framed, graphicx, hyperref, mathrsfs}

\title{\textbf{10月22日作业}}
\author{韩岳成 524531910029}
\date{\today}
\linespread{1.5}
\definecolor{shadecolor}{RGB}{241, 241, 255}
\newcounter{problemname}
\newenvironment{problem}{\begin{shaded}\stepcounter{problemname}\par\noindent\textbf{题目\arabic{problemname}. }}{\end{shaded}\par}
\newenvironment{solution}{\par\noindent\textbf{解答. }}{\par}
\newenvironment{note}{\par\noindent\textbf{题目\arabic{problemname}的注记. }}{\par}

\begin{document}

\maketitle
\begin{problem}
设$f(x)\in C^2([a,b])$,且$f(a)=f(b)=0$,求证$\max_{a\le x\le b}|f(x)|\le\frac18(b-a)^2\max_{a\le x\le b}|f''(x)|$.
\end{problem}

\begin{solution}
若$\max_{a\le x\le b}|f(x)|$出现在端点,则显然有$f(x)\equiv 0, f''(x)\equiv 0$,此时不等式成立。

若$\max_{a\le x\le b}|f(x)|$不出现在端点,设$c\in(a,b)$使得$|f(c)|=\max_{a\le x\le b}|f(x)|$,假设$f(c)>0$(若$f(c)<0$,则可以考虑$-f(x)$),由极值条件可得$f'(c)=0$.

由拉格朗日余项形式的二阶泰勒展开式,对于任意 $x\in[a,b]$,存在 $\xi$ 介于 $x$ 与 $c$ 之间,使得
\[
f(x)=f(c)+f'(c)(x-c)+\frac{1}{2}f''(\xi)(x-c)^2.
\]
当 $x=a$ 时,利用 $f(a)=0$ 及 $f'(c)=0$,得
\[
0=f(c)+\frac{1}{2}f''(\xi_1)(a-c)^2,
\]
其中 $\xi_1\in(a,c)$。因此
\[
f(c)=-\frac{1}{2}f''(\xi_1)(a-c)^2,
\]
从而
\[
|f(c)|\le \frac{1}{2}\max_{[a,b]}|f''(x)|\,(c-a)^2.
\]

同理,对 $x=b$,存在 $\xi_2\in(c,b)$ 使
\[
|f(c)|\le \frac{1}{2}\max_{[a,b]}|f''(x)|\,(b-c)^2.
\]

于是
\[
|f(c)|\le \frac{1}{2}\max_{[a,b]}|f''(x)|\min\{(c-a)^2,(b-c)^2\}.
\]

由于对任意 $c\in[a,b]$,都有
\[
\min\{(c-a)^2,(b-c)^2\}\le \left(\frac{b-a}{2}\right)^2,
\]
当且仅当 $c=\frac{a+b}{2}$ 时取等号。由此得到
\[
|f(c)|\le \frac{1}{2}\cdot\frac{(b-a)^2}{4}\max_{[a,b]}|f''(x)|
=\frac{1}{8}(b-a)^2\max_{[a,b]}|f''(x)|.
\]

因为 $|f(c)|=M=\max_{a\le x\le b}|f(x)|$,故
\[
\max_{a\le x\le b}|f(x)|\le\frac{1}{8}(b-a)^2\max_{a\le x\le b}|f''(x)|.
\]
\end{solution}

\begin{problem}
$f(x)=x^7+x^4+3x+1,\text{求 }f[2^0,2^1,\cdots,2^7]$及$f[2^0,2^1,\cdots,2^8]$.
\end{problem}

\begin{solution}
将$f(x)$写成Newton插值多项式的形式:\\
$f(x)=N_n(x)=f(2^0)+ f[x_0,x_1](x-2^0)+ \cdots + f[2^0,2^1,\cdots,2^7]\\(x-2^0)(x-2^1)\cdots(x-2^6)+f[2^0,2^1,\cdots,2^8](x-2^0)(x-2^1)\cdots(x-2^7)\\=f[2^0,2^1,\cdots,2^8]x^8+[(-2^0-2^1-\cdots-2^7)f[2^0,2^1,\cdots,2^8]+f[2^0,2^1,\cdots,2^7]]x^7+\cdots\\=x^7+x^4+3x+1$

将系数对比可得:
\[f[2^0,2^1,\cdots,2^7]=1\]
\[f[2^0,2^1,\cdots,2^8]=0\]
\end{solution}

\begin{problem}
求一个次数不高于四次的多项式$P(x)$,使它满足$P(0)=P'(0)=0, P(1)=P'(1)=1, P(2)=1.$
\end{problem}

\begin{solution}
对$P(0)=P'(0)=0,P(1)=P'(1)=1$,可构造出两点三次Hermite插值多项式:
\[H_3(x) = 0 + 1 \times (1 + 2\frac{x - 1}{0 - 1})(\frac{x - 0}{1 - 0})^2 + 0 + 1 \times (x - 1)(\frac{x - 0}{1 - 0})^2=(2-x)x^2\]
因为$x^2(x-1)^2$在$x=0,1$处及其导数均为0,所以设$P(x)=H_3(x)+Ax^2(x-1)^2$,代入$P(2)=1$可得:
\[P(2)=(2-2)2^2+A\times2^2(2-1)^2=A\times4=1\Rightarrow A=\frac14\]
所以所求多项式为:
\[P(x)=(2-x)x^2+\frac14x^2(x-1)^2=\frac14x^2(x-3)^2\]
\end{solution}
\end{document}

