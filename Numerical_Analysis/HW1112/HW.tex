\documentclass[12pt, a4paper, oneside]{ctexart}
\usepackage{amsmath, amsthm, amssymb, bm, color, framed, graphicx, hyperref, mathrsfs}

\title{\textbf{11月12日作业}}
\author{韩岳成 524531910029}
\date{\today}
\linespread{1.5}
\definecolor{shadecolor}{RGB}{241, 241, 255}
\newcounter{problemname}
\newenvironment{problem}{\begin{shaded}\stepcounter{problemname}\par\noindent\textbf{题目\arabic{problemname}. }}{\end{shaded}\par}
\newenvironment{solution}{\par\noindent\textbf{解答. }}{\par}
\newenvironment{note}{\par\noindent\textbf{题目\arabic{problemname}的注记. }}{\par}

\begin{document}

\maketitle

\begin{problem}
    求$f(x)=e^x$在$[0,1]$上的一次最佳一致逼近多项式。
\end{problem}

\begin{solution}
    \[c_1 = \frac{f(1)-f(0)}{1-0}=e-1\]
    令$f'(t_2)=c_1$, 则$e^{t_2}=e-1$, 解得$t_2=\ln(e-1)$.因此\[p_1(x)=\frac{f(0)+f(t_2)}{2}+c_1(x-\frac{0+t_2}{2})=\frac{e}{2}+(e-1)(x-\frac{\ln(e-1)}{2})\]
\end{solution}

\begin{problem}
设$f(x)=x^4+3x^3-1$,在$[0,1]$上求三次最佳一致逼近多项式。
\end{problem}

\begin{solution}
所求的三次最佳一致逼近多项式应满足
\[ \max_{0\leq x \leq 1} (f(x)-p_3^*(x)) = \inf_{p_3\in H_3}\max_{0\leq x \leq 1} |f(x)-p_3(x)| \]
令
\[x=\frac12 t+\frac 12, t\in[-1,1]\]
则
\[f(x)=f(\frac12 t+\frac 12)=\frac{1}{16}t^4+\frac{5}{8}t^3+\frac{3}{2}t^2+\frac{11}{8}t-\frac{9}{16}\]
根据Chebyshev首一多项式$\frac{1}{2^{n-1}}T_n(t)$与零的偏差最小的定理可得,当
\[f(x)-p_3^*(t)=\frac{1}{16\times 2^3}T_4(t)=\frac{1}{128}(8t^4-8t^2+1)\]
时,$p_3^*(t)$为所求的三次最佳一致逼近多项式。解得
\[p_3^*(t)=\frac{5}{8}t^3+\frac{25}{16}t^2+\frac{11}{8}t-\frac{73}{128}\]
将$t$换回$x$,得
\[p_3^*(x) = 5x^3-\frac{5}{4}x^2+\frac{1}{4}x-\frac{129}{128}\]
\end{solution}

\begin{problem}
$f(x)=|x|$在[-1,1]上,求在$\varphi_1=\mathrm{span}\langle1,x^2,x^4\rangle$上的最佳平方逼近.
\end{problem}

\begin{solution}
设所求的最佳平方逼近多项式为
\[p(x)=a_0+a_2x^2+a_4x^4\]
则
\[\begin{cases}
\int_{-1}^1 (|x|-p(x))\mathrm{d}x=0\\
\int_{-1}^1 (|x|-p(x))x^2\mathrm{d}x=0\\
\int_{-1}^1 (|x|-p(x))x^4\mathrm{d}x=0
\end{cases}\]
计算必要的矩:
\[\int_{-1}^1 |x|\mathrm{d}x=1, \quad \int_{-1}^1 |x|x^2\mathrm{d}x=\frac{1}{2}, \quad \int_{-1}^1 |x|x^4\mathrm{d}x=\frac{1}{3}\]
再计算基函数的内积:
\[\int_{-1}^1 1\cdot 1 \mathrm{d}x=2, \quad \int_{-1}^1 1\cdot x^2 \mathrm{d}x=\frac{2}{3}, \quad \int_{-1}^1 1\cdot x^4 \mathrm{d}x=\frac{2}{5}\]
\[\int_{-1}^1 x^2\cdot x^2 \mathrm{d}x=\frac{2}{5}, \quad \int_{-1}^1 x^2\cdot x^4 \mathrm{d}x=\frac{2}{7}, \quad \int_{-1}^1 x^4\cdot x^4 \mathrm{d}x=\frac{2}{9}\]
代入上式,得方程组
\[\begin{pmatrix}2&\frac23&\frac25 \\\frac23&\frac25&\frac27\\\frac25&\frac27&\frac29\end{pmatrix}\begin{pmatrix}a_0\\a_2\\a_4\end{pmatrix}=\begin{pmatrix}1\\\frac12\\\frac13\end{pmatrix}\]
解得
\[a_0=\frac{15}{128}, \quad a_2=\frac{105}{64}, \quad a_4=-\frac{105}{128}\]
因此所求的最佳平方逼近多项式为
\[p(x)=\frac{15}{128}+\frac{105}{64}x^2-\frac{105}{128}x^4\]
\end{solution}

\end{document}