\documentclass[12pt, a4paper, oneside]{ctexart}
\usepackage{amsmath, amsthm, amssymb, bm, color, framed, graphicx, hyperref, mathrsfs}

\title{\textbf{10月22日作业}}
\author{韩岳成 524531910029}
\date{\today}
\linespread{1.5}
\definecolor{shadecolor}{RGB}{241, 241, 255}
\newcounter{problemname}
\newenvironment{problem}{\begin{shaded}\stepcounter{problemname}\par\noindent\textbf{题目\arabic{problemname}. }}{\end{shaded}\par}
\newenvironment{solution}{\par\noindent\textbf{解答. }}{\par}
\newenvironment{note}{\par\noindent\textbf{题目\arabic{problemname}的注记. }}{\par}

\begin{document}

\maketitle

\begin{problem}
    % 题目内容
    随机变量X服从参数为0.6的$(0,1)$分布, 在{X=0}及{X=1}的条件下随机变量Y的条件分布律如下:
    \begin{center}
    \begin{tabular}{c|ccc}
    \hline
    $Y$ & 1 & 2 & 3 \\
    \hline
    $P(Y \mid X = 0)$ & 0.25 & 0.5 & 0.25 \\
    \hline
    \end{tabular}
    \quad
    \begin{tabular}{c|ccc}
    \hline
    $Y$ & 1 & 2 & 3 \\
    \hline
    $P(Y \mid X = 1)$ & $\frac{1}{2}$ & $\frac{1}{6}$ & $\frac{1}{3}$ \\
    \hline
    \end{tabular}
    \end{center}
    求在{$Y=1$}以及{$Y\neq1$}的条件下随机变量X的条件分布律.
\end{problem}

\begin{solution}
    由于随机变量X服从参数为0.6的$(0,1)$分布, 故$P(X=0)=0.4$, $P(X=1)=0.6$.
    由全概率公式, 可得
    \begin{equation}
        \begin{aligned}
            P(Y=1) &= P(Y=1 \mid X=0)P(X=0) + P(Y=1 \mid X=1)P(X=1) \\
            &= 0.25 \times 0.4 + \frac{1}{2} \times 0.6 = 0.4.
        \end{aligned}
        \nonumber
    \end{equation}
    \[P(Y\neq1) = 1 - P(Y=1) = 0.6.\]
    因此, 在{$Y=1$}的条件下, 随机变量X的条件分布律为
    \begin{equation}
        \begin{aligned}
            P(X=0 \mid Y=1) &= \frac{P(Y=1 \mid X=0)P(X=0)}{P(Y=1)} = \frac{0.25 \times 0.4}{0.4} = 0.25, \\
            P(X=1 \mid Y=1) &= \frac{P(Y=1 \mid X=1)P(X=1)}{P(Y=1)} = \frac{\frac{1}{2} \times 0.6}{0.4} = 0.75.
        \end{aligned}
        \nonumber
    \end{equation}
    在{$Y\neq1$}的条件下, 
    \[P(Y\neq1 \mid X=0) = 1 - P(Y=1 \mid X=0) = 0.75,\]
    \[P(Y\neq1 \mid X=1) = 1 - P(Y=1 \mid X=1) = \frac{1}{2}.\]
    因此,随机变量X的条件分布律为
    \begin{equation}
        \begin{aligned}
            P(X=0 \mid Y\neq1) &= \frac{P(Y\neq1 \mid X=0)P(X=0)}{P(Y\neq1)} = \frac{0.75 \times 0.4}{0.6} = 0.5, \\
            P(X=1 \mid Y\neq1) &= \frac{P(Y\neq1 \mid X=1)P(X=1)}{P(Y\neq1)} = \frac{\frac{1}{2} \times 0.6}{0.6} = 0.5.
        \end{aligned}
        \nonumber
    \end{equation}
\end{solution}

\begin{problem}
    在$n$重 Bernoulli 试验中,若事件$A$出现的概率为$p$,令
    \[X_i=\begin{cases}1,&\text{在第}i\text{次试验中}A\text{发生,}\\0,&\text{在第}i\text{次试验中}A\text{ 不发生},\end{cases}i=1,2,\cdots,n\],

    求在$\{X_1+X_2+\cdots+X_n=r\}\left(0\leqslant r\leqslant n\right)$的条件下$X_i\left(0\leqslant i\leqslant n\right)$的条件分布律.
\end{problem}

\begin{solution}
    \begin{equation}
        \begin{aligned}
            &P(X_i=1 | X_1+X_2+\cdots+X_n=r) \\
            &= \frac{P(X_i=1, X_1+X_2+\cdots+X_n=r)}{P(X_1+X_2+\cdots+X_n=r)} \\
            &= \frac{p \times C_{n-1}^{r-1} p^{r-1} (1-p)^{n-r}}{C_n^r p^r (1-p)^{n-r}} \\
            &= \frac{r}{n}, \\
            &P(X_i=0 | X_1+X_2+\cdots+X_n=r) \\
            &= 1 - P(X_i=1 | X_1+X_2+\cdots+X_n=r) \\
            &= 1 - \frac{r}{n} = \frac{n-r}{n}.
        \end{aligned}
        \nonumber
    \end{equation}
\end{solution}

\begin{problem}
设随机变量 $(X,Y)$ 的联合概率密度为
$$f(x,y) = \begin{cases}
\frac{x^3}{2}e^{-x(1+y)}, & x>0, y>0, \\
0, & \text{其他},
\end{cases}$$
求:(1) $(X,Y)$ 关于 $X$ 的边缘概率密度 $f_X(x)$;
(2) 条件概率密度 $f_{Y|X}(y|x)$,并写出当 $x=0.5$ 时的条件概率密度;
(3) 条件概率 $P(Y\geqslant 1|X=0.5)$.
\end{problem}

\begin{solution}
(1) 边缘概率密度:
\begin{equation}
    \begin{aligned}
        f_X(x) &= \int_{-\infty}^{+\infty} f(x,y) dy = \int_0^{+\infty} \frac{x^3}{2} e^{-x(1+y)} dy \\
        &= \frac{x^3}{2} e^{-x} \int_0^{+\infty} e^{-xy} dy = \frac{x^3}{2} e^{-x} \cdot \frac{1}{x} = \frac{x^2}{2} e^{-x}, \quad x>0.
    \end{aligned}
    \nonumber
\end{equation}
因此\[f_X(x) = \begin{cases}\frac{x^2}{2} e^{-x}, & x>0, \\ 0, & \text{其他}.\end{cases}\]

(2) 条件概率密度:
\begin{equation}
    \begin{aligned}
        f_{Y|X}(y|x) &= \frac{f(x,y)}{f_X(x)} = \frac{\frac{x^3}{2} e^{-x(1+y)}}{\frac{x^2}{2} e^{-x}} = x e^{-xy}, \quad y>0.
    \end{aligned} 
    \nonumber
\end{equation}
因此$f_{Y|X}(y|x) = \begin{cases} x e^{-xy}, & y>0, \\ 0, & \text{其他}.\end{cases}$. 

当$x=0.5$时, $f_{Y|X}(y|0.5) = \begin{cases} 0.5 e^{-0.5y}, & y>0, \\ 0, & \text{其他}.\end{cases}$.

(3) 条件概率:
\begin{equation}
    \begin{aligned}
        P(Y \geqslant 1 | X=0.5) &= \int_1^{+\infty} f_{Y|X}(y|0.5) dy = \int_1^{+\infty} 0.5 e^{-0.5y} dy \\
        &= 0.5 \left[-2 e^{-0.5y}\right]_1^{+\infty} = e^{-0.5}.
    \end{aligned}
    \nonumber
\end{equation}
\end{solution}

\begin{problem}
设(X,Y)是二维随机变量,其关于X的边缘概率密度为
\[f_{X}(x)=\begin{cases}\frac{2+x}{6}, & 0<x<2, \\0, & \text { 其他 },\end{cases}\]
且当$\{X=x\}(0<x<2)$时Y的条件概率密度为
$$f_{Y \mid X}(y \mid x)=\begin{cases}
\frac{1+xy}{1+x/2}, & 0<y<1, \\
0, & \text { 其他. }
\end{cases}$$
求:\,\,\,\,(1)(X,Y)的联合概率密度;

(2)(X,Y)关于Y的边缘概率密度;

(3)在$\{Y=y\}$的条件下X的条件概率密度$f_{X \mid Y}(x \mid y)$.
\end{problem}

\begin{solution}
(1)\begin{equation}
    \begin{aligned}
        f(x,y) &= f_X(x) f_{Y|X}(y|x) = \frac{2+x}{6} \cdot \frac{1+xy}{1+x/2} \\
        & = \frac{1+xy}{3}, \quad\quad 0<x<2, 0<y<1.
    \end{aligned}
    \nonumber
\end{equation}
因此,(X,Y)的联合概率密度为
\[f(x,y) = \begin{cases} \frac{1+xy}{3}, & 0<x<2, 0<y<1,\\ 0, & \text{其他}.\end{cases}\]

(2)\begin{equation}
    \begin{aligned}
        f_Y(y) &= \int_{-\infty}^{+\infty} f(x,y) dx = \int_0^2 \frac{1+xy}{3} dx, \quad 0<y<1.
    \end{aligned}
    \nonumber
\end{equation}
计算可得
\[f_Y(y) = \begin{cases} \frac{2+2y}{3}, & 0<y<1, \\ 0, & \text{其他}.\end{cases}\]

(3)\begin{equation}
    \begin{aligned}
        f_{X|Y}(x|y) &= \frac{f(x,y)}{f_Y(y)} = \frac{\frac{1+xy}{3}}{\frac{2+2y}{3}} = \frac{1+xy}{2+2y}, \quad 0<x<2.
    \end{aligned}
    \nonumber
\end{equation}
因此,
\begin{equation}
    f_{X|Y}(x|y) = \begin{cases} \frac{1+xy}{2+2y}, & 0<x<2, \\ 0, & \text{其他}.\end{cases}
\nonumber
\end{equation}
\end{solution}

\begin{problem}
设随机变量 \( X, Y \) 满足 \( P(XY=0)=1 \),且其分布律分别如下表所示:

\[
\begin{array}{c|ccc}
X & -1 & 0 & 2 \\
\hline
P_i & \frac{1}{4} & \frac{1}{2} & \frac{1}{4} \\
\end{array}
\]

\[
\begin{array}{c|cc}
Y & 0 & 1 \\
\hline
P_i & \frac{1}{2} & \frac{1}{2} \\
\end{array}
\]

(1) 求在 \( \{X=0\} \) 的条件下 \( Y \) 的条件分布律;

(2) 试问 \( X \) 与 \( Y \) 是否相互独立?为什么?
\end{problem}

\begin{solution}
由题意可知, $P(XY\neq 0)= 0$, 所以$P(X=-1, Y=1)=0$且$P(X=2, Y=1)=0$.

因此(X,Y)的联合分布律如下表所示:

\[
\begin{array}{|c|ccc|c|}
\hline
P_{ij}& X=-1 & X=0 & X=2 & P_j = \sum_i P_{ij} \\
\hline
Y=0 & \frac{1}{4} & 0 & \frac{1}{4} & \frac{1}{2} \\
Y=1 & 0 & \frac{1}{2} & 0 & \frac{1}{2} \\
\hline
P_i = \sum_j P_{ij} & \frac{1}{4} & \frac{1}{2} & \frac{1}{4} & \\
\hline
\end{array}
\]
(1) 由全概率公式, 可得
\[P(Y=0 | X=0) = \frac{P(X=0, Y=0)}{P(X=0)} = \frac{0}{\frac{1}{2}} = 0,\]
\[P(Y=1 | X=0) = \frac{P(X=0, Y=1)}{P(X=0)} = \frac{\frac{1}{2}}{\frac{1}{2}} = 1.\]
因此,在 \( \{X=0\} \) 的条件下 \( Y \) 的条件分布律为:
\[\begin{array}{c|cc}
Y & 0 & 1 \\
\hline
P(Y | X=0) & 0 & 1 \\
\end{array}\]
(2) 由于$P(Y=0|X=0)\neq P(Y=0)$, 故随机变量X与Y不相互独立.
\end{solution}

\begin{problem}
尝试用多种方法解下面的题目:

假设随机变量$X\sim U(0,1)$,并且当$X=x(0 < x < 1)$时,随机变量$Y$服从$(0, x)$上的均匀分布,即$Y\sim U(0, x)$,并且随机变量$Y$的可能取值范围为$(0, 1)$,求$Y$的分布函数。
\end{problem}

\begin{solution}
\begin{itemize}
    \item[方法一:]因为$X\sim U(0,1)$,所以$X$的概率密度函数为
\begin{equation}
    f_X(x)=\begin{cases}
    1, &0<x<1;\\
    0, &\text{其他}.
    \end{cases}
    \nonumber
\end{equation}
由于随机变量Y服从$(0,x)$上的均匀分布, 故在$\{X=x\}(0<x<1)$时, Y的条件概率密度为
\begin{equation}
    f_{Y|X}(y|x)=\begin{cases}
    \frac1x,&0<y<x;\\
    0,&\text{其他}.
    \end{cases}
    \nonumber
\end{equation}
因此,
\begin{equation}
    \begin{aligned}
        f_Y(y) &= \int_{-\infty}^{+\infty} f_{X,Y}(x,y) dx\\
        &= \int_{0}^{1} f_{Y|X}(y|x) f_X(x) dx\\
        &= \int_y^1 \frac{1}{x} \cdot 1 \, dx,\\
        &= [-\ln x]_y^1 = -\ln y.\quad 0<y<1. 
    \end{aligned}
    \nonumber
\end{equation}
\begin{equation}
    \begin{aligned}
    F_Y(y)=&\int_0^y f_Y(y)dy\\
    &= \int_0^y -\ln t \, dt\\
    &= y - y \ln y, \quad 0<y<1.
    \end{aligned}
    \nonumber
\end{equation}
因为分布函数为
\begin{equation}
    F_Y(y)=\begin{cases}
    0, & y\leqslant 0;\\
    y - y \ln y, & 0<y<1;\\
    1, & y\geqslant 1.
    \end{cases}
    \nonumber
\end{equation}
\item[方法二:]已知随机变量 $X\sim U(0,1)$,并且在给定 $X=x$ 时,
\[
Y \mid X=x \sim U(0,x).
\]
因此,对任意固定的 $x$(其中 $0<x<1$),有条件分布函数
\[
P(Y \le y \mid X=x) =
\begin{cases}
0, & y \le 0,\\[6pt]
\dfrac{y}{x}, & 0<y<x,\\[6pt]
1, & y \ge x.
\end{cases}
\]

因为 $X\sim U(0,1)$,其密度为 $f_X(x)=1\ (0<x<1)$。
由全概率公式可得,对 $0<y<1$,
\[
\begin{aligned}
F_Y(y)
&= P(Y \le y)
= \int_0^1 P(Y \le y \mid X=x)\,f_X(x)\,dx \\[6pt]
&= \int_0^y 1 \, dx + \int_y^1 \frac{y}{x}\,dx \\[6pt]
&= y + y\int_y^1 \frac{1}{x}\,dx \\[6pt]
&= y - y\ln y.
\end{aligned}
\]

于是 $Y$ 的分布函数为
\[
F_Y(y) =
\begin{cases}
0, & y \le 0,\\[6pt]
y(1-\ln y), & 0<y<1,\\[6pt]
1, & y \ge 1.
\end{cases}
\]

\end{itemize}
\end{solution}

\begin{problem}
设$(X,Y)\sim N\left(0,4;2,9;0\right)$,求

(1)$P\left (|X|> 1\right ) ;$

(2)$f_{Y\mid X}\left ( y\mid x\right ) ;$ 

(3)$P\left ( X< 3\mid Y= 2\right ) .$
\end{problem}

\begin{solution}
(1) 由正态分布的性质可知,$X$的分布为$N(0,4)$,因此
\begin{equation}
    \begin{aligned}
        P(|X|>1) &= P(X<-1) + P(X>1) \\
        &= 2P(X>1)\\
        &= 2(1-\Phi(\frac{1-0}{2}))\\
        &= 0.617
    \end{aligned}
    \nonumber
\end{equation}
(2) 由正态分布中$\rho = 0$可得,随机变量$X$与$Y$相互独立,因此
\begin{equation}
    f_{Y|X}(y|x) = f_Y(y) = \frac{1}{3\sqrt{2\pi}} e^{-\frac{(y-2)^2}{18}}.
    \nonumber
\end{equation}
(3) 由随机变量$X$与$Y$相互独立可知
\begin{equation}
    P(X<3 | Y=2) = P(X<3) = \Phi(\frac{3-0}{2}) = 0.9332.
    \nonumber
\end{equation}
\end{solution}
\end{document}