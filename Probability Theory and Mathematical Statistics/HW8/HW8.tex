\documentclass[12pt, a4paper, oneside]{ctexart}
\usepackage{amsmath, amsthm, amssymb, bm, color, framed, graphicx, hyperref, mathrsfs}

\title{\textbf{10月22日作业}}
\author{韩岳成 524531910029}
\date{\today}
\linespread{1.5}
\definecolor{shadecolor}{RGB}{241, 241, 255}
\newcounter{problemname}
\newenvironment{problem}{\begin{shaded}\stepcounter{problemname}\par\noindent\textbf{题目\arabic{problemname}. }}{\end{shaded}\par}
\newenvironment{solution}{\par\noindent\textbf{解答. }}{\par}
\newenvironment{note}{\par\noindent\textbf{题目\arabic{problemname}的注记. }}{\par}

\begin{document}

\maketitle

\begin{problem}
    % 题目内容
    随机变量X服从参数为0.6的$(0,1)$分布, 在{X=0}及{X=1}的条件下随机变量Y的条件分布律如下:
    \begin{center}
    \begin{tabular}{c|ccc}
    \hline
    $Y$ & 1 & 2 & 3 \\
    \hline
    $P(Y \mid X = 0)$ & 0.25 & 0.5 & 0.25 \\
    \hline
    \end{tabular}
    \quad
    \begin{tabular}{c|ccc}
    \hline
    $Y$ & 1 & 2 & 3 \\
    \hline
    $P(Y \mid X = 1)$ & $\frac{1}{2}$ & $\frac{1}{6}$ & $\frac{1}{3}$ \\
    \hline
    \end{tabular}
    \end{center}
    求在{$Y=1$}以及{$Y\neq1$}的条件下随机变量X的条件分布律.
\end{problem}

\begin{solution}
    由于随机变量X服从参数为0.6的$(0,1)$分布, 故$P(X=0)=0.4$, $P(X=1)=0.6$.
    由全概率公式, 可得
    \begin{equation}
        \begin{aligned}
            P(Y=1) &= P(Y=1 \mid X=0)P(X=0) + P(Y=1 \mid X=1)P(X=1) \\
            &= 0.25 \times 0.4 + \frac{1}{2} \times 0.6 = 0.4.
        \end{aligned}
        \nonumber
    \end{equation}
    \[P(Y\neq1) = 1 - P(Y=1) = 0.6.\]
    因此, 在{$Y=1$}的条件下, 随机变量X的条件分布律为
    \begin{equation}
        \begin{aligned}
            P(X=0 \mid Y=1) &= \frac{P(Y=1 \mid X=0)P(X=0)}{P(Y=1)} = \frac{0.25 \times 0.4}{0.4} = 0.25, \\
            P(X=1 \mid Y=1) &= \frac{P(Y=1 \mid X=1)P(X=1)}{P(Y=1)} = \frac{\frac{1}{2} \times 0.6}{0.4} = 0.75.
        \end{aligned}
        \nonumber
    \end{equation}
    在{$Y\neq1$}的条件下, 
    \[P(Y\neq1 \mid X=0) = 1 - P(Y=1 \mid X=0) = 0.75,\]
    \[P(Y\neq1 \mid X=1) = 1 - P(Y=1 \mid X=1) = \frac{1}{2}.\]
    因此,随机变量X的条件分布律为
    \begin{equation}
        \begin{aligned}
            P(X=0 \mid Y\neq1) &= \frac{P(Y\neq1 \mid X=0)P(X=0)}{P(Y\neq1)} = \frac{0.75 \times 0.4}{0.6} = 0.5, \\
            P(X=1 \mid Y\neq1) &= \frac{P(Y\neq1 \mid X=1)P(X=1)}{P(Y\neq1)} = \frac{\frac{1}{2} \times 0.6}{0.6} = 0.5.
        \end{aligned}
        \nonumber
    \end{equation}
\end{solution}

\begin{problem}
    在$n$重 Bernoulli 试验中,若事件$A$出现的概率为$p$,令
    \[X_i=\begin{cases}1,&\text{在第}i\text{次试验中}A\text{发生,}\\0,&\text{在第}i\text{次试验中}A\text{ 不发生},\end{cases}i=1,2,\cdots,n\],

    求在$\{X_1+X_2+\cdots+X_n=r\}\left(0\leqslant r\leqslant n\right)$的条件下$X_i\left(0\leqslant i\leqslant n\right)$的条件分布律.
\end{problem}

\begin{solution}
    \begin{equation}
        \begin{aligned}
            &P(X_i=1 | X_1+X_2+\cdots+X_n=r) \\
            &= \frac{P(X_i=1, X_1+X_2+\cdots+X_n=r)}{P(X_1+X_2+\cdots+X_n=r)} \\
            &= \frac{p \times C_{n-1}^{r-1} p^{r-1} (1-p)^{n-r}}{C_n^r p^r (1-p)^{n-r}} \\
            &= \frac{r}{n}, \\
            &P(X_i=0 | X_1+X_2+\cdots+X_n=r) \\
            &= 1 - P(X_i=1 | X_1+X_2+\cdots+X_n=r) \\
            &= 1 - \frac{r}{n} = \frac{n-r}{n}.
        \end{aligned}
        \nonumber
    \end{equation}
\end{solution}
\end{document}