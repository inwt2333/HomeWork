\documentclass[12pt, a4paper, oneside]{ctexart}
\usepackage{amsmath, amsthm, amssymb, bm, color, framed, graphicx, hyperref, mathrsfs}

\title{\textbf{11月12日作业}}
\author{韩岳成 524531910029}
\date{\today}
\linespread{1.5}
\definecolor{shadecolor}{RGB}{241, 241, 255}
\newcounter{problemname}
\newenvironment{problem}{\begin{shaded}\stepcounter{problemname}\par\noindent\textbf{题目\arabic{problemname}. }}{\end{shaded}\par}
\newenvironment{solution}{\par\noindent\textbf{解答. }}{\par}
\newenvironment{note}{\par\noindent\textbf{题目\arabic{problemname}的注记. }}{\par}

\begin{document}

\maketitle

\begin{problem}
    设二维随机变量的联合分布律为
    \begin{center}
    \includegraphics[width=0.8\textwidth]{image_1.png}
    \end{center}
    试验证X和Y既不相关,也不相互独立。
\end{problem}

\begin{solution}
    计算得到边缘分布律为
    \[ P(X=-1) = \frac{3}{8}, \quad P(X=0) = \frac{1}{4}, \quad P(X=1) = \frac{3}{8} \]
    \[ P(Y=-1) = \frac{3}{8}, \quad P(Y=0) = \frac{1}{4}, \quad P(Y=1) = \frac{3}{8} \]
    计算$E(X), E(Y), E(XY)$:
    \[
        E(X) = (-1) \cdot \frac{3}{8} + 0 \cdot \frac{1}{4} + 1 \cdot \frac{3}{8} = 0
    \]
    \[
        E(Y) = (-1) \cdot \frac{3}{8} + 0 \cdot \frac{1}{4} + 1 \cdot \frac{3}{8} = 0
    \]
    \[
        E(XY) = 1 \cdot 2 \cdot \frac{1}{8} + 0 \cdot 4 \cdot \frac{1}{8} + (-1) \cdot 2 \cdot \frac{1}{8} = 0
    \]
    因为$E(XY) = E(X)E(Y) = 0$,所以X和Y不相关。

    再验证X和Y是否相互独立,取$P(X=1, Y=1) = \frac{1}{8}$,而$P(X=1)P(Y=1) = \frac{3}{8} \cdot \frac{3}{8} = \frac{9}{64}$,所以X和Y不相互独立。
\end{solution}

\begin{problem}
设A, B是试验E的两个随机事件, 且P(A)>0, P(B)>0, 并定义随机变量X与Y如下:
$$X= \begin{cases}
1, & A \text { 发生, } \\
0, & \bar{A} \text { 发生, }
\end{cases}
\quad
Y= \begin{cases}
1, & B \text { 发生, } \\
0, & \bar{B} \text { 发生, }
\end{cases}$$
证明: 若X与Y不相关, 则X与Y必定相互独立.
\end{problem}

\begin{solution}
    若X与Y不相关,则$E(XY) = E(X)E(Y)$, 又有$E(X) = P(A)$, $E(Y) = P(B)$, $E(XY) = P(AB)$,所以$P(AB) = P(A)P(B)$,即A与B相互独立。
\end{solution}

\begin{problem}
    设二维随机变量$(X,Y)\sim N(1,9;0,16;-0.5)$,令$Z=\frac{X}{3}+\frac{Y}{2}$,求$E(Z),D(Z),\rho_{XZ}$.
\end{problem}

\begin{solution}
    由$Z=\frac{X}{3}+\frac{Y}{2}$,可得
    \[
        E(Z) = E\left(\frac{X}{3}+\frac{Y}{2}\right) = \frac{E(X)}{3} + \frac{E(Y)}{2} = \frac{1}{3} + 0 = \frac{1}{3}
    \]
    \[
        D(Z) = D\left(\frac{X}{3}+\frac{Y}{2}\right) = \left(\frac{1}{3}\right)^2 D(X) + \left(\frac{1}{2}\right)^2 D(Y) + 2 \cdot \frac{1}{3} \cdot \frac{1}{2} \cdot Cov(X,Y)
    \]
    其中$Cov(X,Y) = \rho_{XY} \sqrt{D(X)D(Y)} = -0.5 \cdot \sqrt{9 \cdot 16} = -6$,所以
    \[
        D(Z) = \frac{1}{9} \cdot 9 + \frac{1}{4} \cdot 16 + 2 \cdot \frac{1}{3} \cdot \frac{1}{2} \cdot (-6) = 1 + 4 - 2 = 3
    \]
    接下来计算$\rho_{XZ}$:
    \[
        Cov(X,Z) = Cov\left(X, \frac{X}{3} + \frac{Y}{2}\right) = Cov\left(X, \frac{X}{3}\right) + Cov\left(X, \frac{Y}{2}\right) = \frac{1}{3} D(X) + \frac{1}{2} Cov(X,Y)
    \]
    所以
    \[
        Cov(X,Z) = \frac{1}{3} \cdot 9 + \frac{1}{2} \cdot (-6) = 3 - 3 = 0
    \]
    因此
    \[
        \rho_{XZ} = \frac{Cov(X,Z)}{\sqrt{D(X)D(Z)}} = \frac{0}{\sqrt{9 \cdot 3}} = 0
    \]
\end{solution}

\begin{problem}
设$(X,Y)\sim N(2,9;2,4;0)$,如果相互独立对 $(X,Y)$ 进行观察,$Z_1$表示直到出现 $X>Y$ 为止所需要的观察次数,$Z_2$表示10次观察中出现 $X>Y$ 的次数。求$Z_1, Z_2$的数学期望与方差.
\end{problem}

\begin{solution}
    由于$(X,Y)$相互独立,故$X-Y \sim N(0, 9+4) = N(0, 13)$,令$Z = X - Y$,则
    \[
        P(X>Y) = P(X-Y > 0) = P\left(Z > 0\right) = P\left(\frac{Z-0}{\sqrt{13}} > 0\right) = \frac{1}{2}
    \]
    其中$Z \sim N(0,13)$。

    对于$Z_1$,它服从参数为$p=\frac{1}{2}$的几何分布,所以
    \[
        E(Z_1) = \frac{1}{p} = 2, \quad D(Z_1) = \frac{1-p}{p^2} = 2
    \]

    对于$Z_2$,它服从参数为$n=10, p=\frac{1}{2}$的二项分布,所以
    \[
        E(Z_2) = np = 10 \cdot \frac{1}{2} = 5, \quad D(Z_2) = np(1-p) = 10 \cdot \frac{1}{2} \cdot \frac{1}{2} = \frac{5}{2}
    \]
\end{solution}

\begin{problem}
 将$n$个球随机地放入 $n$ 个盒子中,每盒容纳的球数无限,求空着的盒子数的数学期望和方差.
\end{problem}

\begin{solution}
    设随机变量$X$表示空着的盒子数,定义指示变量
    \[
        X_i = \begin{cases}
        1, & \text{第}i\text{个盒子为空} \\
        0, & \text{第}i\text{个盒子不为空}
        \end{cases}
    \]
    则$X = \sum_{i=1}^{n} X_i$。计算$E(X_i)$:
    \[
        E(X_i) = P(X_i = 1) = \left(1 - \frac{1}{n}\right)^n
    \]
    因此
    \[
        E(X) = E\left(\sum_{i=1}^{n} X_i\right) = \sum_{i=1}^{n} E(X_i) = n \left(1 - \frac{1}{n}\right)^n
    \]

    接下来计算$D(X)$,首先计算$E(X^2)$:
    \[
        E(X^2) = E\left(\left(\sum_{i=1}^{n} X_i\right)^2\right) = E\left(\sum_{i=1}^{n} X_i^2 + 2\sum_{1 \leq i < j \leq n} X_i X_j\right)
    \]
    注意到$X_i^2 = X_i$,所以
    \[
        E(X^2) = \sum_{i=1}^{n} E(X_i) + 2\sum_{1 \leq i < j \leq n} E(X_i X_j)
    \]
    计算$E(X_i X_j)$:
    \[
        E(X_i X_j) = P(X_i = 1, X_j = 1) = \left(1 - \frac{2}{n}\right)^n
    \]
    因此
    \[
        E(X^2) = n\left(1 - \frac{1}{n}\right)^n + n(n-1)\left(1 - \frac{2}{n}\right)^n
    \]
    最后,方差为
    \[
        D(X) = E(X^2) - (E(X))^2 = n\left(1 - \frac{1}{n}\right)^n + n(n-1)\left(1 - \frac{2}{n}\right)^n - n^2\left(1 - \frac{1}{n}\right)^{2n}
    \]
\end{solution}
\end{document}