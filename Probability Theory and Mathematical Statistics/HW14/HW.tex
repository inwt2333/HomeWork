\documentclass[12pt, a4paper, oneside]{ctexart}
\usepackage{amsmath, amsthm, amssymb, bm, color, framed, graphicx, hyperref, mathrsfs}

\title{\textbf{11月19日作业}}
\author{韩岳成 524531910029}
\date{\today}
\linespread{1.5}
\definecolor{shadecolor}{RGB}{241, 241, 255}
\newcounter{problemname}
\newenvironment{problem}{\begin{shaded}\stepcounter{problemname}\par\noindent\textbf{题目\arabic{problemname}. }}{\end{shaded}\par}
\newenvironment{solution}{\par\noindent\textbf{解答. }}{\par}
\newenvironment{note}{\par\noindent\textbf{题目\arabic{problemname}的注记. }}{\par}

\begin{document}

\maketitle

\begin{problem}
    某电力公司供应其所在地区7500户居民用电,假设各户用电情况相互独立,并且每户每日用电量(单位:$kW\cdot h$)在$[0,20]$上服从均匀分布,请利用中心极限定理估计:

    (1)这7500户居民每日用电总量超过76000$kW\cdot h$的概率;

    (2)要以99.9\%的概率保证该地区居民用电的需求,该电力公司每天至少需向该地区供应多少电量?
\end{problem}

\begin{solution}
    设每一户居民的每日用电量为随机变量$X_i$,则$X_i \sim U(0,20)$,且相互独立,则$E(X_i)=10, D(X_i)=\frac{100}{3}$。
    
    设总用电量为$X = \sum_{i=1}^{7500} X_i$,则$E(X) = 75000, D(X) = 7500 \times \frac{100}{3} = 250000 = 500^2$。因此可以近似认为$X\sim(75000,500^2)$。

    (1) $P(X>76000) =1 - \Phi(\frac{76000-75000}{500}) = 1-\Phi(2) \approx 0.0228$;

    (2) 设电力公司每天至少需供应电量为$Q$,则有
    $$P(X \leq Q) = 0.999.$$
    根据正态分布的性质,
    $$\frac{Q - 75000}{500} = \Phi^{-1}(0.999) \approx 3.09,$$
    解得
    $$Q \approx 75000 + 3.09 \times 500 = 76545.$$

\end{solution}

\begin{problem}
    据统计数据可知某商店出售一种贵重商品,每周销售量(单位:件)的分布律如下:

    \centering
    \begin{tabular}{c|ccc}
    \hline
    $X$ & 0 & 1 & 2 \\
    \hline
    $P$ & 0.2 & 0.6 & 0.2 \\
    \hline
    \end{tabular}

    \raggedright
    假定每周销售量相互独立。

    (1) 用 Chebyshev 不等式估计一年(按照 52 周算)的累积销售量在 42 到 62 之间的概率;

    (2) 用中心极限定理估计一年累积销售量在 42 到 62 之间的概率.
\end{problem}

\begin{solution}
设每周销售量为$X_i$,可计算得$E(X_i) = 1, D(X_i) = 0.4$,设一年累积销售量为$X = \sum_{i=1}^{52} X_i$,则$E(X) = 52, D(X) = 52 \times 0.4 = 20.8$。

(1) 由Chebyshev多项式可得
\[P(42\le X \le 62) = P(|X - E(X)|)< 10 = 1 - \frac{D(X)}{10^2} = 0.792\]

(2) 我们可以近似认为$X\sim N(52, 20.8)$,则
\[P(42\le X\le 62) = \phi(\frac{62-52}{\sqrt{20.8}}) - \phi(\frac{42-52}{\sqrt{20.8}}) \approx 0.9714\]
\end{solution}

\begin{problem}
    设$\xi_n$为$n$重 Bernoulli 试验中成功的次数$,p(0<p<1)$为每次试验成功的概率,当$n$ 充分大时,$\forall\varepsilon>0$,使用 De Moivre-Laplace 中心极限定理证明
    $$P\bigg(\bigg|\frac{\xi_n}{n}-p\bigg|<\varepsilon\bigg)\approx2\boldsymbol{\Phi}\bigg(\boldsymbol{\varepsilon}\sqrt{\frac{n}{p(1-p)}}\bigg)-1.$$
\end{problem}

\begin{solution}
    由题意知$\xi_n\sim B(n,p)$,则
    \[E(\xi_n)=np, D(\xi_n)=np(1-p).\]
    由 De Moivre-Laplace 中心极限定理可知,当n较大时,可近似认为
    \[\frac{\xi_n-np}{\sqrt{np(1-p)}} \sim N(0,1),\]
    \begin{align*}
        P\bigg(\bigg|\frac{\xi_n}{n}-p\bigg|<\varepsilon\bigg)&=P\bigg(\frac{|\xi_n-np|}{n}<\varepsilon\bigg)\\&=P\bigg(\frac{|\xi_n-np|}{\sqrt{np(1-p)}}<\varepsilon\sqrt{\frac{n}{p(1-p)}}\bigg)\\
        &\approx\Phi\left(\varepsilon\sqrt{\frac{n}{p(1-p)}}\right) - \Phi\left(-\varepsilon\sqrt{\frac{n}{p(1-p)}}\right)\\
        &= 2\Phi\left(\varepsilon\sqrt{\frac{n}{p(1-p)}}\right) - 1
    \end{align*}
\end{solution}
\begin{problem}
    假设某便利店每天接待顾客数按200位计, 每位顾客的消费额 X (单位: 元) 的概率密度为
    $$f(x)=\left\{\begin{array}{ll}k(30-|x-30|), & x \in[0,60], \\0, & \text { 其他, }\end{array}\right.$$
    且各顾客的消费额相互独立.

    (1) 求 k;

    (2) 用 Chebyshev 不等式估计该便利店每天的营业额在 5800 到 6200 元之间的概率.

    (3) 用中心极限定理估计该便利店每天的营业额在 5800 到 6200 元之间
\end{problem}

\begin{solution}
    (1) 
    \[\int_{-\infty}^{+\infty}f(x)=\int_{0}^{30}kx\mathrm{d}x+\int_{30}^{60}k(60-x)\mathrm{d}x = 900k = 1\]
    解得$k=\frac{1}{900}$.

    (2)\[E(X)=\int_{0}^{60}xf(x)\mathrm{d}x = 30\]
    \[E(X^2)=\int_0^{60}x^2f(x)\mathrm{d}x = 1050, \quad D(X)=E(X^2)-E(X)^2 = 150\]
    设总利润$Y=\sum X$,则$E(Y) = 6000, D(Y) = 30000$.

    由Chebyshev多项式可得
    \[P(5800<Y<6200)=P(|Y-E(Y)|<200) = 1-\frac{D(Y)}{200^2} = 0.25\]

    (3)我们可以近似认为$Y\sim N(6000, 30000)$,则
    \[P(5800<Y<6200)=\Phi(\frac{6200-6000}{\sqrt{30000}}) - \Phi(\frac{5800-6000}{\sqrt{30000}}) = 2P(\frac{2}{\sqrt{3}})-1\approx 0.7498\]
\end{solution}

\begin{problem}
    设随机变量序列 $X_1, X_2, \cdots$ 相互独立且同分布 $B(m, p)$,记 $\overline{X_n} = \frac{1}{n} \sum_{k=1}^{n} X_k$,当 n 充分大时求 $\overline{X_n}$ 的近似分布。
\end{problem}

\begin{solution}
由题意知,由于$X_i\sim B(m, p)$,则$\sum_{k=1}^n X_k \sim B(mn,p)$。

当n充分大时,mn充分大,可近似认为$\sum_{k=1}^n X_k \sim N(mnp, mnp(1-p))$, 因此
\[\overline{X_n} = \frac{1}{n} \sum_{k=1}^n X_k \sim N\left(m p, \frac{m p (1-p)}{n}\right).\]
\end{solution}

\begin{problem}
尝试用概率论的方法证明:

$$\lim_{n\to\infty}(1+n+\frac{n^2}{2!}+\cdots+\frac{n^n}{n!})e^{-n}=\frac12$$
\end{problem}

\begin{solution}
假设$X_n\sim P(n)$,则$P(X_n=k)=e^{-n}\frac{n^k}{k!}\quad(k=0,1,2,\cdots)$,因此
\[ S_n=\sum_{k=0}^{n}e^{-n}\frac{n^k}{k!}=P(X_n\le n)\]

由于$E(X_n)= D(X_n)=n$,利用中心极限定理,令$Z_n=\frac{X_n-n}{\sqrt{n}}$,则当n足够大时,可近似认为$Z_n\sim N(0,1)$。

因此$S_n=P(X_n\le n)=P(Z_n\le 0)=\Phi(0)=\frac{1}{2}$.即
$$\lim_{n\to\infty}(1+n+\frac{n^2}{2!}+\cdots+\frac{n^n}{n!})e^{-n}=\frac12$$
\end{solution}
\end{document}